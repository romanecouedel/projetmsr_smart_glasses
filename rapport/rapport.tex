\documentclass[a4paper,12pt]{report}

% Encodage et langue
\usepackage[utf8]{inputenc}
\usepackage[T1]{fontenc}
\usepackage[french]{babel}

% Mise en page
\usepackage{geometry}
\geometry{
    left=2.5cm,
    right=2.5cm,
    top=2.5cm,
    bottom=2.5cm
}

% Packages
\usepackage{graphicx}
\usepackage{hyperref}
\usepackage{lipsum}
\usepackage{tikz}
\usetikzlibrary{shapes, arrows, positioning}   
\usepackage{everypage} % Pour ajouter le logo sur toutes les pages
\usepackage{longtable}% Pour les tableaux longs

% Fonction pour mettre le logo sur toutes les pages
\newcommand{\AddLogo}{
    \begin{tikzpicture}[remember picture, overlay]
        \node[anchor=north west, xshift=0.75cm, yshift=-0.75cm] 
        at (current page.north west) {\includegraphics[width=3cm]{logo_su.jpg}};
    \end{tikzpicture}
}

% Ajouter le logo à chaque page
\AddEverypageHook{\AddLogo}

\begin{document}

% Page de garde
\begin{titlepage}
    \centering
    \vspace*{4cm}
    {\Huge \textbf{Smart Glasses}}\\[1.5cm]
    {\Large Authors : Sarah Nacim Evinia Oualid Romane}\\[0.5cm]
    {\Large Master 2 : MSR 2025-2026}\\[3cm]
    \includegraphics[width=0.8\textwidth]{logo.png}% 
\end{titlepage}

% Table des matières
\tableofcontents
\newpage

% Chapitres
\section{Introduction}

\subsection{Project Context}

With the rise of wearable technology, smart glasses are undergoing significant development. Companies such as Meta, with their Ray-Ban Meta glasses, are now offering devices capable of capturing images and video, interacting via voice commands, and accessing AI-based assistants.

At the same time, within the field of assistive technology, several research projects and commercial products aim to support visually impaired or blind individuals specifically through obstacle detection, scene description, and the reading of visual information. Furthermore, recent breakthroughs in artificial intelligence, particularly in natural language processing (NLP) and computer vision, are opening up new possibilities to enhance visual assistance tools.

\subsection{Objectives}

In this context, our project follows a similar path, but with a specific focus on social interaction and assistance. The objective is to design smart glasses capable of detecting both the emotions of the person facing the user and any obstacles within the environment.

However, to ensure reliable operation and straightforward interaction, these two features do not run simultaneously. The system is organized into different operating modes, accessible through an interface of physical buttons integrated into the glasses. This allows the user to switch between "obstacle detection" and "emotion detection" modes, depending on their immediate needs.

This approach simplifies data processing, improves the clarity of haptic feedback, and provides the user with greater control over the device's behavior.

\section{Materials Used}

\begin{longtable}{|p{3cm}|p{2cm}|p{3cm}|p{6cm}|}
\hline
\textbf{Component} & \textbf{Number} & \textbf{Reference / Model} & \textbf{Description / Role} \\
\hline
\endfirsthead

\hline
\textbf{Component} & \textbf{Number} & \textbf{Reference / Model} & \textbf{Description / Role} \\
\hline
\endhead

\hline
\endfoot

\hline
\endlastfoot

Camera & 1 & Camera Raspberry module 2 & Sony IMX219 8-megapixel sensor \\
\hline
Raspberry Pi & 1 & Raspberry Pi 3B+ & Used to process image data and send it to the computer \\
\hline
Arduino & 1 & Arduino Uno & Used to control the haptic feedback system (ie pneumatic actuators...) \\
\hline
Differential Pressure sensor & 2 & DFRobot SEN0343 & Used to detect pressure into balloons \\
\hline
Motor driver & 3 & L298N & Used to control 2 motors per driver \\
\hline
Valves & 2 & DFR0866 & Used to switch between inflate and deflate \\
\hline
Pumps & 4 & DFRobot 370 Mini Vacuum Pump & 2 used to inflate and 2 to deflate balloons \\
\hline
Balloons & 2 & Standard party balloons & Used to do haptic feedback \\
\hline
Push buttons & 2 & Standard push buttons & Used to switch between modes \\
\hline
Computer & 1 & PC linux & Used to run the AI models for emotion and obstacle detection \\
\hline
Battery 12V & 1 & lead acid battery 12V & Used to power the pumps \\
\hline
Battery 5V and <2.5A & 2 & Standard Power bank 5V & Used to power the Raspberry Pi and Arduino \\
\hline

\caption{Main hardware components used in the project}
\end{longtable}


\section{Implementation}
\subsection{Overall Architecture}

\begin{center}
\begin{tikzpicture}[
    block/.style={rectangle, draw, rounded corners, minimum width=3.5cm, minimum height=1cm, align=center},
    arrow/.style={->, thick}
]

% Nodes
\node[block] (user) {User};
\node[block, right=5cm of user] (arduino) {Arduino};
\node[block, above=2cm of user] (raspi) {Raspberry Pi};
\node[block, right=5cm of raspi] (computer) {Computer / AI Models};
% Arrows
\draw[arrow] (raspi.east) -- (computer.west) node[midway, above] {Camera Stream (WIFI)};
\draw[arrow] (computer.south) -- (arduino.north east) node[midway, right] {Control signals};
\draw[arrow] (user.north east) -- (arduino.north west) node[midway, above] {Button press};
\draw[arrow] (arduino.south west) -- (user.south east) node[midway, above] {Haptic feedback};
\draw[arrow] (arduino.north) -- (computer.south west) node[midway, above, sloped] {Mode info};

\end{tikzpicture}
\end{center}

\subsection{Camera and Raspberry pi}
\subsection{Pneumatic/haptic system}
\subsection{Computer and AI models}
\subsection{User Interface}


\section{Fonctionnalités}


\section{Objectifs initiaux et limites du projet}

\section{Pistes d'amélioration}

\end{document}
